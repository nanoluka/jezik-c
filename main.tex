\documentclass{beamer}
%\usepackage{pgfpages}
%\pgfpagesuselayout{4 on 1}[a4paper,border shrink=5mm]
\usepackage[utf8]{inputenc}
\usepackage[serbianc]{babel}
\usepackage{textgreek}
\usepackage{pgfpages}

\usetheme{Boadilla}
\usecolortheme[named=orange]{structure}
\usepackage{caption}
\usepackage{subcaption}
\usepackage[final]{pdfpages}
\usepackage{xcolor}

\usepackage{listings}



\useoutertheme[footline=authortitle]{miniframes}

\title[Програмски језик C - основи]{\color{black}\smallУвод у \\\color{orange}\normalsizeПрограмски језик C}
\vspace{-4cm}\author{Александар Пајкановић}
\titlegraphic{\includegraphics[width=2cm]{logo}}

%\titlegraphic{\includegraphics[width=2cm]{logo}%\hspace{10.75cm}%

%}
\institute{\texttt{nanoluka.org}\\
YouTube.com/nanoluka\\
instagram.com/nanolukaorg\\
twitter.com/nanolukaorg\\
github.com/nanoluka\\
\texttt{nanolukaorg@gmail.com}}

\date{\color{orange}Октобар 2020}

\begin{document}

\maketitle

\begin{frame}{Кориштење овог документа}
\begin{itemize}
    \item \textbf{Лиценца}: 
    \begin{itemize}
        \item Документ објављујем под лиценцом \color{orange}\href{https://creativecommons.org/licenses/by-sa/4.0//legalcode}{\textit{Attribution-ShareAlike — CC BY-СА}}\color{black}. Ова лиценца дозвољава ремикс и прераду, као и комерцијално кориштење дјела, ако/док се \textbf{правилно назначава име аутора} и ако се \textbf{прерада лиценцира под истим условима}. Ова лиценца се често упоређује са „\textit{copyleft}” лиценцирањем слободног софтвера или софтвера отвореног к\^{о}да. Сва дјела настала на основу дјела лиценцираног овом лиценцом, требало би да буду лиценцирана истом лиценцом, која, поред осталог, дозвољава комерцијално кориштење.
    \end{itemize}
    \item \textbf{Навођење (цитирање)}: 
    \begin{itemize}
        \item Ако сте користили овај документ као извор у сопственим материјалима, молим да цитирате на сљедећи начин: А. Пајкановић, ,,Увод у програмски језик \textit{C} кроз практчне примјере'', доступно на \color{orange}\href{github.com/nanoluka/jezik-c.git}{github.com/nanoluka/jezik-c.git}
    \end{itemize}
\end{itemize}
\end{frame}



\begin{frame}{Програмирање и програмски језици}
\begin{itemize}
    \item Рачунар у општем смислу је машина способна да изврши било који алгоритам пратећи дефинисан скуп правила - ткз. Тјурингова машина (теоријска замисао).
    \item Алгоритам је недвосмислен метод рјешавања конкретног проблема
    \item Рачунар се већ читав вијек састоји од процесора и меморије - ткз. Фон Нојманова архитектура (реализација)
    \item Програмирање, у практичном смислу, можемо рећи да представља састављање текстуалног (понегдје и графичког) описа жељеног понашања рачунара у смислу обраде корисничких података и представљања резултата тог процеса кориснику
    \item Програмски језик је скуп инструкција (кључних ријечи) које, ако су посложене у складу са унапријед дефинисаним правилима, рачунар може да прихвати - при чему ово ,,прихвати'' значи да је у стању да, узев дате податке корисника на улазу, кориснику прикаже очекивани излаз
\end{itemize}
\end{frame}

\begin{frame}{Програмски језик C}
\begin{itemize}
    \item Постоји много подјела програмских језика, најважнија је на:
    \begin{itemize}
        \item језике вишег нивоа, и
        \item језике нижег нивоа
    \end{itemize}
    \item \textit{C} спада у језик вишег нивоа. Генерално, сви програмски језици за које сте чули су, у ствари, у истој групи. Ова друга група своди се на машински, тј. асемблер - о том потом, неки други пут.
    \item Друга важна подјела је према парадигми којој програмски језик припада, тренутно су актуелне четири парадигме (које се, често, међусобно пресјецају):
    \begin{itemize}
        \item императивно програмирање,
        \item функционално,
        \item логичко, и
        \item објектно оријентисано.
    \end{itemize}
    \item Наш данашњи језик спада у прву групу, а његов млађи брат \textit{C++} у посљедњу.
\end{itemize}
\end{frame}

\begin{frame}{Програмски језик C}
\begin{itemize}
    \item Иначе, \textit{C} је насљедник програмског језика \textit{B} (гле чуда), настао је почетком 70их у Бел Лабораторијама, човјек се звао Денис Ричи.
    \item Најважније карактеристике су веома ефикасан превод у асемблерске инструкције, писан је за извршавање на различитим платформама, брзина извршавања, велика слобода, па је зато веома распрострањен - данас највише у домену програмирања уграђених система.
    \item С друге стране, за разлику од пајтона, на примјер, који је можда најраспрострањенији на планети, крива учења је ипак нешто стрмија.
    \item Укратко, поступак од идеје до реализације састоји се од пет корака.
\end{itemize}
\end{frame}

\begin{frame}{Програмски језик C}
\begin{itemize}
    \item Укратко, поступак од идеје до реализације састоји се од пет корака:
    \begin{enumerate}
        \item алгоритам,
        \item псеудо к\^{о}д,
        \item изворни к\^{о}д - овдје се пише синтакса (\texttt{.c}),
        \item компајлирањем добијамо објектни к\^{о}д (\texttt{.obj}), и
        \item линкер нам даје реализацију, ткз. извршни к\^{о}д, односно извршну датотеку (\texttt{.exe})
    \end{enumerate}
    \item На том путу, сусрећемо се са разним невољама, а подијелићемо их у грешксе:
    \begin{itemize}
        \item синтаксичке, и
        \item семантичке.
    \end{itemize}
    \item Ова презентација не представља потпуне информације о рачунарима, програмирању и програмском језику \textit{C}, него служи као увод или преглед, како бисмо се упознали са основама и знали гдје и шта да тражимо како бисмо стварно научили.
\end{itemize}
\end{frame}

\begin{frame}{Потребни алати}
\begin{itemize}
    \item Рачунар - бар током овог предавања, али може и телефон или таблет - само је унос проблематичан
    \item \textit{Windows} оперативни систем - може, наравно, и неки други али данас рад демонстрирамо овако
    \item \textit{Dev C++} - интегрисано развојно окружење (енгл. \textit{IDE}), потпун алат и лак за инсталацију и кориштење, бесплатно доступан на \color{orange}\href{https://sourceforge.net/projects/orwelldevcpp/}{https://sourceforge.net/projects/orwelldevcpp/}
    \item \color{black}\textit{Arduino} - интегрисано развојно окружење за рад са Ардуино платформом, пакет са свим потенцијалним додацима доступан на: \color{orange}\href{https://www.arduino.cc/}{https://www.arduino.cc/}\color{black}
    \item Познавање енглеског језика
    \item Добра воља, чврста одлука да се не одустаје и упоран рад
\end{itemize}
\end{frame}

\begin{frame}[fragile]{Примјер 1 - Испис текста}
\begin{block}{HelloICBL}
\begin{lstlisting}
#include <stdio.h>
int main() {

   // printf() displays the string inside quotation
   printf("Hello, ICBL!");
   return 0;
}
\end{lstlisting}
\end{block}

\end{frame}

\begin{frame}{Типови података}
\begin{itemize}
    \item Основна подјела сигнала:
    \begin{itemize}
        \item аналогни - звук, бежични пренос података, зрачење, слика
        \item дигитални - запис аналогних, али користећи само ограничен број нивоа - ако су само два нивоа, онда су то бинарни сигнали
    \end{itemize}
    \item Данашњи комерцијално доступни рачунари су дигитални и разумију искључиво бинарне податке
    \item Математички запис је дат прије два вијека, данас познат као бинарна или Булова алгебра
    \item Једна бинарна цифра назива се бит (енгл. \textit{binary digit} $\rightarrow$ \textit{bit})
    \item Осам бита је бајт (енгл. \textit{byte}), kilo, mega, итд.
    \item Подаци су осмобитни, 32-битни, итд.
    \item Корисни записи су још и октални и хексадецимални
    \item Децимални број 6, осмобитно се представља као \texttt{00000110}, број 126 пишемо \texttt{01111110}, а -126 je \texttt{10000010}
    \item Означенио и неозначени
\end{itemize}
\end{frame}

\begin{frame}[fragile]{Примјер 2 - Типови података}
\begin{block}{DataTypes}
\begin{lstlisting}
#include <stdio.h>      
int main() {
  int a;
  long b;
  char c;
  float d;

  printf("size of int = %d bytes\n", sizeof(a));
  printf("size of short = %d bytes\n", sizeof(b));
  printf("size of char = %d bytes\n", sizeof(c));
  printf("size of float = %d bytes\n", sizeof(d));
  return 0;
}
\end{lstlisting}
\end{block}
\end{frame}

\begin{frame}[fragile]{Примјер 3 - Аритметичке и логичке операције}
\begin{block}{ALU}
\begin{lstlisting}
  int a = 2;
  int b = 4;
  char c = 'a';
  float d = 6.2;

  printf("zbir: a+b = %d\n", a+b);
  printf("razlika: a-b = %d\n", a-b);
  printf("proizvod: a*c = %d\n", a*c);
  printf("kolicnik - cjelobrojno: a/b= %f\n", a/b);
  printf("kolicnik - decimalno: a/d= %f\n", a/b);
  printf("jednakost: a == b = %d\n", a==b);
  printf("binarno I: a & b = %d\n", a & b);
  printf("logicko I: a && b = %d\n", a && b);
\end{lstlisting}
\end{block}
\end{frame}

\begin{frame}{Контрола т\^{о}ка}
\begin{itemize}
    \item Рачунари не одлучују. Бар не још увијек
    \item Рачунари поступају по инструкцијама
    \item Инструкција може да се изврши или не изврши у зависности од услова, односно може да се изврши једна или друга, опет, наравно, зависно од стања нечег другог:
    \begin{itemize}
        \item ако је температура већа од 22$^o$ C, искључи гријање
        \item ако је температура мања од 18$^o$ C, укључи гријање
        \item не извршавај ништа, док се не притисне овај тастер
        \item изврши сабирање свих бројева у овом низу, тј. 10 000 сабирања
    \end{itemize}
    \item Графички приказ т\^{о}ка извршавања назива се дијаграм т\^{о}ка
    \item Т\^{о}к контролишемо користећи се гранањем (\texttt{if-else, switch}) и петљама (\texttt{for, while, do-while})
\end{itemize}
\end{frame}

\begin{frame}[fragile]{Примјер 4 - Унос података}
\begin{block}{DataInput}
\begin{lstlisting}
#include <stdio.h>
int main()
{
    int a;
    printf("Unesite broj: ");
    scanf("%d", &a);  
    
    printf("Broj je = %d", a);
    return 0;
}
\end{lstlisting}
\end{block}
\end{frame}

\begin{frame}[fragile]{Примјер 5 - Гранање}
\begin{block}{Branch}
\begin{lstlisting}
    int a;
    printf("Unesite broj: ");
    scanf("%d", &a);  
    
    if (a > 9) {
        printf("Broj %d je dvocifren", a);}
    else if (a > 100) {
        printf("Broj %d je trocifren", a);}
    else {
        printf("Broj %d je jednocifren", a);}
\end{lstlisting}
\end{block}
\end{frame}

\begin{frame}[fragile]{Примјер 6 - Петља}
\begin{block}{Loop}
\begin{lstlisting}
    int a;
    int f = 1;
    printf("Unesite broj: ");
    scanf("%d", &a);  
    
    int i;
    for(i = 0; i<a; i++){
        printf("Ispisujem %d. put\n", i);
        f = f * (i+1);
        }
    printf("Faktorijel broja %d je: %d\n", a, f);
\end{lstlisting}
\end{block}
\end{frame}

\begin{frame}{Енкапсулација и апстракција}
\begin{itemize}
    \item Тако далеко сам видио, зато што сам стојао сам на плећима дивова
    \item Све што постоји, а да је створено људском руком, настало је комбинацијом већ постојећег
    \item Зашто да измишљамо топлу воду?
    \item Дајте ми полугу, помјерићу свијет
    \item Право питање је: шта је проблем?
    \item Видјели смо који су типови података, видјели смо које су доступне аритметичке операције.
    \item Како ћемо израчунати коријен? Експонент? Синус? Интеграл?
\end{itemize}
\end{frame}

\begin{frame}[fragile]{Примјер 7 - Кориштење библиотека}
\begin{block}{Library / Lib}
\begin{lstlisting}
#include <math.h>
#include <stdio.h>

int main() {
    float a = 3, e = 4, r = 0;

    // calculates the power
    r = pow(a, e);

    printf("%.1lf^%.1lf = %.2lf", a, e, r);
    return 0;
}
\end{lstlisting}
\end{block}
\end{frame}

\begin{frame}[fragile]{Примјер 8 - Писање функција}
\begin{block}{Function}
\begin{lstlisting}
#include <stdio.h>
float stepen(float x, float y) {
    int i;
    float r = 1;
    for(i = 1; i <= y; i++) {
        r = r * x; }
    return r; }
int main() {
    float a = 3, e = 4, r = 0;
    r = stepen(a, e);
    printf("%.1lf^%.1lf = %.2lf", a, e, r);
    return 0; }
\end{lstlisting}
\end{block}
\end{frame}

\begin{frame}{Уграђени (енгл. \textit{embedded}) системи}
\begin{itemize}
    \item Уграђени системи су рачунарски системи намијењени за извршавање специфичне функције у реалном времену, и могу да буду сасставни дио обимнијих електромеханичких (над)система.
    \item Рачунари су, на примјер, системи опште намјене, дакле нису ,,уграђени'' у овом смислу.
    \item Махом су дигитални, а централни дио је увијек микроконтролер.
    \item Микроконтролер је процесор са периферијама и (малом) меморијом.
    \item Internet of Things - интернет свега, паметни системи
    \item Аутомобилска индустрија (енгл. \textit{automotive})
    \item Интересантан свијет, на граници између хардвера и софтвера, рачунарска електроника, рачунарски инжењеринг, мехатроника, роботика.
    \item Ардуино је сјајан први корак на том путу.
\end{itemize}
\end{frame}

\begin{frame}[fragile]{Примјер 9 - Трепћући Ардуино}
\begin{block}{Blink}
\begin{lstlisting}
void setup() {
  pinMode(LED_BUILTIN, OUTPUT);     //izlazni
}
void loop() /*PETLJA*/ {
  digitalWrite(LED_BUILTIN, HIGH);  //VISOK nivo
  delay(1000);                      //sacekaj
  digitalWrite(LED_BUILTIN, LOW);   //NIZAK nivo
  delay(1000);                      //sackeaj
}
\end{lstlisting}
\end{block}
\end{frame}

\begin{frame}[fragile]{Примјер 10 - Зглоб једног робота}
\begin{block}{Motor}
\begin{lstlisting}
digitalWrite(smjer,HIGH); // Izbor smjera rotacije
for(int x = 0; x < 200; x++) {
    digitalWrite(stepPin,HIGH); 
    delayMicroseconds(500); 
    digitalWrite(stepPin,LOW); 
    delayMicroseconds(500); }
delay(1000); // Sacekaj jednu sekundu
digitalWrite(smjer,LOW); // Promjena smjera rotacije
for(int x = 0; x < 400; x++) {
    digitalWrite(stepPin,HIGH);
    delayMicroseconds(500);
    digitalWrite(stepPin,LOW);
    delayMicroseconds(500); }
 
 delay(1000); // Sacekaj opet
\end{lstlisting}
\end{block}
\end{frame}

\begin{frame}{Референце}
\begin{itemize}
    \item \color{orange}\href{https://www.google.com/}{Google}!!!
    \item \color{orange}\href{https://skolakoda.github.io/kursevi/ucimo-c/}{Увод у програмирање кроз \textit{C}}, \color{black} Школа к\^{o}да
    \item \color{orange}\href{https://www.programiz.com/}{Programiz}
    \item \color{orange}\href{https://www.coursera.org/}{Coursera} \color{black}и \color{orange}\href{https://www.edx.org/}{edX}
    \item \color{orange}\href{http://uticnionica.etf.unibl.org/}{Утичнионица}\color{black} - основно о Ардуину, са Електротехничког факултета у Бањој Луци
    \item Упутство за рад са \textit{CNC} наставком за Ардуино, \color{orange}\href{https://www.youtube.com/watch?v=TMK_fLgpESQ}{видео} \color{black}и \color{orange}\href{https://www.electroniclinic.com/arduino-cnc-shield-v3-0-and-a4988-hybrid-stepper-motor-driver-joystick/}{текст}
    \item \color{orange}\href{http://1000zabuducnost.org/}{1000 за будућност}\color{black} - увод у рад са:
    \begin{itemize}
        \item Пи рачунар
        \item програмски језик \textit{Python}
    \end{itemize}
    \item Званична презентација \color{orange}\href{https://www.raspberrypi.org/}{\textit{Raspberry Pi}}\color{black}
    \item Увод у електронику - серија \color{orange}\href{https://www.youtube.com/playlist?list=PLh-StTZA7RZ6Ch6Esin2mnoPzFvLZOYu3}{Три минута електронике}
    \item \href{https://www.nanoluka.org}{Књига о LTspice}\color{black}, софтверском симулатору електричних кола
\end{itemize}
\end{frame}

\end{document}